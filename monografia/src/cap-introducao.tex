%% ------------------------------------------------------------------------- %%
\chapter{Introdução}
\label{cap:introducao}

O impacto negativo de sistemas computacionais no meio ambiente é evidente. Como Murugesan nota em \cite{murugesan2008harnessing}, cada fase do ciclo de vida de um computador implica em diferentes problemas ambientais. A Computação Verde combina política, ciências, e engenharia para tentar resolver esses problemas.

Um dos ramos da computação verde é o uso eficiente de recursos computacionais. Para estudar este ramo, é interessante dividí-lo nas camadas \emph{hardware}, sistema operacional, aplicação, e redes como Ardito faz em sua tese de doutoramento \cite{ardito2014energy}. As camadas de \emph{hardware} e sistema operacional foram as mais estudadas por serem de mais baixo nível e, portanto, mais diretamente relacionadas com o consumo energético.

\vspace{5mm}

O foco deste trabalho é na camada de aplicação, que só recentemente passou a ser explorada. Fizemos 4 experimentos sobre esta camada com 2 objetivos principais:
\begin{enumerate}
    \item Expandir e validar resultados de outros pesquisadores para o nosso sistema.
    \item Encontrar oportunidades de aumentar a eficiência energética.
\end{enumerate}

\vspace{5mm}
Tanto os capítulos quanto suas subdivisões seguem uma sequência lógica. Assim, para melhor compreensão do trabalho, sugerimos a leitura na ordem apresentada. Porém, como os capítulos de revisão são superficiais, o leitor familiarizado com os conceitos poderá focar diretamente no experimentos sem maiores problemas. A organização dos capítulos é a seguinte:
\begin{enumerate}
    \item Pequena revisão sobre computação verde e eficiência energética nos Capítulos 2 e 3.
    \item Apresentação de nosso sistema de medição no Capítulo 4.
    \item Especificação dos experimentos e análises dos seus resultados no Capítulo 5.
    \item Recapitulação e conclusão no capítulo 6.
\end{enumerate}


%% ------------------------------------------------------------------------- %%
%\section{Contribuições}
%\label{sec:contribucoes}
%
%As principais contribuições deste trabalho são as seguintes:
%
%\begin{itemize}
%  \item
%

%\end{itemize}

%\section{Organização do Trabalho}
%\label{sec:organizacao_trabalho}

% No Capítulo~\ref{cap:conceitos}, apresentamos os conceitos ... Finalmente, no
% Capítulo~\ref{cap:conclusoes} discutimos algumas conclusões obtidas neste
% trabalho. Analisamos as vantagens e desvantagens do método proposto ...

% As sequências testadas no trabalho estão disponíveis no Apêndice \ref{ape:sequencias}.
