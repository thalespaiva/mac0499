\documentclass[a4paper,11pt]{article} % Formato do papel, tipo de documento e tamanho da fonte.
\usepackage[brazilian]{babel}
\usepackage[utf8]{inputenc}
%\usepackage[brazil]{babel} % Hifenização em português
\usepackage[T1]{fontenc} % Caracteres com acentos são considerados como um bloco
\usepackage{ae} % Arruma a fonte quando usa o pacote acima
\usepackage{amssymb} % Caracteres matemáticos especiais
\usepackage[pdftex]{graphicx} % Para inserir figuras
\usepackage{titlesec}
\usepackage[section]{placeins}
\usepackage{listings}


%\usepackage{graphicx}
%\renewcommand{\theenumi}{\Alph{enumi}}
%\setcounter{section}{-1}

%\titleformat{\chapter}{\normalfont\LARGE}{}{20pt}{\LARGE\textbf}



\title{
	\textbf{
        MAC0499 - Avaliação Subjetiva
    }
}

\author	{
	\bf{Thales Paiva - 7156390}     \\
	thalespaiva@gmail.com       \\
}

\date{1/12/2014}
\begin{document}
\maketitle
\tableofcontents
\pagebreak


\section{Escolha do tema}
\mbox{}

Minha ideia inicial era escolher um tema fora do que já conhecesse e de preferência novo. Assim, poderia ter menos trabalho técnico e mais leituras e estudos exploratórios. Lembrei de um dos trabalhos pedidos pelo Professor Alfredo na disciplina de Organização de Computadores de 2013 em que, pela primeira vez, se falava de consumo energético implicado por aplicações. Assim, tínhamos um tema muito recente e bem difícil de ser estudado e minha carga de leitura acabou ficando imensa no primeiro semestre.

Infelizmente\footnote{Para mim, é claro.}, grande parte dos artigos que tratam de eficiência energética na camada de aplicação estudam dispositivos móveis ou embarcados, não PC's e servidores. Assim, como minha proposta original era baseada em artigos que havia lido antes de ter acesso à máquina de testes, muitas das técnicas não valeriam para nosso PC na UFSCar. Por isso, acabei tendo que alterar a proposta original.

Por sorte, foi muito mais divertido de trabalhar sobre o novo tema, que é bem menos mecânico que o primeiro e tem valor para diversas áreas.


\section{Disciplinas mais relevantes para o projeto}
\mbox{}

\subsection{Organização de Computadores}

Como já falamos, foi nesta disciplina em que conheci o problema de eficiência energética na camada de aplicação. Além disso, foi crucial para interpretar os resultados dos experimentos e para conseguir ler os artigos sobre o tema sem grandes dificuldades.

\subsection{Sistemas Operacionais}

Para o design dos experimentos e interpretação dos resultados esta matéria também me ajudou bastante. Infelizmente, o sistema de testes não era nosso e, assim, não pudemos programar para o kernel do sistema.

\subsection{Introdução à Computação}

Programei um pouco em C, e é nesta disciplina que temos o primeiro contato sério com a linguagem. Não poderia deixar de estar aqui.


\subsection{Estruturas de Dados e Análise de Algoritmos}

Nestas matérias, vemos como a escolha das estruturas de dados e dos algoritmos impactam na performance e no consumo de memória. Foram cruciais para a interpretação dos resultados, para o design dos experimentos, para ler os artigos (e para a vida em geral).

\subsection{Laboratório de Programação Extrema}

É nesta disciplina em que, pela primeira vez na graduação, temos que desenvolver um sistema que será realmente utilizado por outras pessoas. Assim, o mesmo espírito que tive nesta disciplina foi o adotado para este trabalho.

\subsection{Programação Concorrente}

Importantíssima para desenhar experimentos, analisar os resultados e ler os artigos. Ainda, há muitas oportunidades para eficiência energética através da programação concorrente.

\subsection{Introdução à Probabilidade e Processos Estocásticos}

Muitos dos modelos para economizar energia com o Hardware e Sistema Operacional são baseados em teoria de processos estocásticos e algoritmos online. Essa disciplina me ajudou bastante a entender os algoritmos e modelos.

\section{Sobre o Resultado do trabalho}
\mbox{}

Fiquei bem satisfeito com os resultados. Consegui estudar um tema novo para mim, e encontrar oportunidades interessantes. Porém, gostaria de ter sido mais rigoroso, ao menos em alguns experimentos. Não acho que será difícil fazer a análise estatística dos dados e alguns dos comentários que tivemos de fazer timidamente (pois não havia validação estatística) poderiam virar afirmações mais embasadas.



\end{document}
