\documentclass[a4paper,11pt]{article} % Formato do papel, tipo de documento e tamanho da fonte.
\usepackage[brazilian]{babel}
\usepackage[utf8]{inputenc}
%\usepackage[brazil]{babel} % Hifenização em português
\usepackage[T1]{fontenc} % Caracteres com acentos são considerados como um bloco
\usepackage{ae} % Arruma a fonte quando usa o pacote acima
\usepackage{amssymb} % Caracteres matemáticos especiais
\usepackage[pdftex]{graphicx} % Para inserir figuras    
\usepackage{titlesec}
\usepackage[section]{placeins}
\usepackage{listings}


%\usepackage{graphicx}
%\renewcommand{\theenumi}{\Alph{enumi}}
%\setcounter{section}{-1}

%\titleformat{\chapter}{\normalfont\LARGE}{}{20pt}{\LARGE\textbf}



\title{
	\textbf{
        MAC0431 - Relatório do EP1
    }
}

\author	{
	\bf{Thales Paiva - 7156390}     \\
	thalespaiva@gmail.com       \\
}

\date{5/12/2014}
\begin{document}
\maketitle
\tableofcontents
\pagebreak

\section{Estrutura do código}

Tentei modularizar o código o máximo possível. Assim, espero deixar o paralelismo evidente nas seções paralelas. Infelizmente, esta abordagem pode fazer (e fez) meu programa perder performance.

\subsection{hexboard.c}

Tratei o tabuleiro usado pelas joaninhas como um tipo abstrato de dados. Assim, qualquer chamada ao tabuleiro deverá seguir a interface:

\begin{verbatim}
int      hexboard_init(int nrows, int ncols);
int      hexboard_cell_is_free(int i, int j);
void     hexboard_declare_intention_to_occupy(int i, int j, double delta_temp);
void     hexboard_cell_set_temp(int i, int j, double temperature);
double   hexboard_cell_get_temp(int i, int j);
double   hexboard_get_distance_squared(int i1, int j1, int i2, int j2);
void     hexboard_occupy_cell(int i, int j);
void     hexboard_free_cell(int i, int j);
void     hexboard_cell_update(int i, int j);
double   hexboard_cell_get_cand_delta_temp(int i, int j);
int      hexboard_cell_get_n_moving_candidates(int i, int j);
int      hexboard_cell_is_null(CellInfo *cell_info);
int      hexboard_cell_rand_r(int i, int j);
void     hexboard_generate_cells_seeds(int *pnt_global_seed);
void     hexboard_cell_put_hf(int i, int j);
void     hexboard_cell_free_hf(int i, int j);
int      hexboard_cell_has_hf(int i, int j);
CellInfo hexboard_get_free_max_temp_neighbour(int i, int j);
CellInfo hexboard_get_free_min_temp_neighbour(int i, int j);
\end{verbatim}

Este módulo não usa a OpenMP em nenhum momento, pois, como as funções de maior consumo só são utilizadas no começo, o código ficaria mais lento.

\subsection{bug.c}

Este módulo contém a implementação da vida de uma joaninha. Ela pode ser inicializada, declarar a intenção de mover para um hexágono vizinho e mover-se para um hexágono vizinho. Assim, as funções deste módulo são:

\begin{verbatim}
Bug * bug_init(double C, double temp_min, double temp_max);
void  bug_declare_intention_to_move(Bug *bug);
void  bug_move(Bug *bug);
\end{verbatim}

\subsection{heatfont.c}

Este módulo contém a implementação das fontes de calor aleatórias. Além de conter funções específicas para uma fonte, contém funções para tratar dos vetores de fonte. Sua interface é:

\begin{verbatim}
HeatFont * heatfont_init(int i, int j, int cycles_to_live, double C);
void       heatfont_add_to_list(HFnode **plist, HeatFont *hf);
int        heatfont_list_update(HFnode **node, HeatFont *heatfonts);
void       heatfont_show(HeatFont *hf);
int        heatfont_add_to_array(HeatFont **heatfonts, HeatFont *hf, int n);
int        heatfont_array_update(HeatFont **heatfonts, int n);
\end{verbatim}

\subsection{engine.c}

Este é o módulo principal e contém o motor da simulação. Além de uma simples interface para que a função principal possa iniciar uma simulação:

\begin{verbatim}
void engine_setup(SetupInfo *info);
void engine_run();
\end{verbatim}

este módulo implementa uma série de funções auxiliares que podem ser paralelizadas. Para mais informações sobre isso, veja a seção sobre paralelização.

\subsection{hotbugs.c}

Contém a função principal.

\pagebreak
\section{Estratégia de paralelização}

Para a paralelização, utilizei o OpenMP. Como ele dá uma interface simples para paralelizar programas sequenciais, primeiro desenvolvi uma versão sequencial e depois paralelizei-a aos poucos. Essa abordagem pode não ser a melhor, pois meu código foi pensado sequencialmente. Porém, pude manter um controle de corretude mínima através da comparação entre os resultados do programa sequencial e dos cada vez mais paralelos.

Em geral, usei a paralelização em funções que usam laços (neste caso, as que mais consomem tempo). Pode-se ver certa paralelização nas funções:
\begin{verbatim}

double calculate_cell_temperature_because_of_bugs(int i, int j);
double calculate_cell_temperature_because_of_heatfonts(int i, int j);
void update_cell_temperature(int i, int j);
void gen_random_heat_fonts();
void update_board();
void move_all_bugs();

\end{verbatim}


\section{Exemplos de entrada e saída}
\mbox{}

Para examinar o comportamento das joaninhas, escrevi um simples (e um tanto precário) script em Python3 para desenhar uma figura representando uma iteração da simulação. Para usá-lo, deve-se fazer:
\begin{verbatim}
./graphs/hexdraw.py ./dumps/dumpcycleXXXX.dat
\end{verbatim}

Neste caso, o diretório dumps/ contém arquivos no formato esperado pelo hexdraw.py. Para este diretório ser preenchido com dumps de cada iteração, basta compilar o programa com a flag {\tt DUMP\_FLAG} valendo 1. Esta flag está no arquivo engine.c, e pode-se alterá-la com:
\begin{verbatim}
#define DUMP_FLAG 1
\end{verbatim}

\subsection{Teste 1}




\section{Aceleração obtida}
\mbox{}


\end{document}
